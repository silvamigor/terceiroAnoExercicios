%%%%%%%%%%%%%%%%%%%%%%%%%%%%%%%%%
%
%       RESPOSTA QUESTÃO DISCURSIVA
%
%%%%%%%%%%%%%%%%%%%%%%%%%%%%%%%%%

\thispagestyle{empty}
\vspace*{1pt}
\backgroundsetup{
  position=current page.center,
  angle=0,
  scale=1,
  opacity=1,
  contents={%
    \begin{tikzpicture}[
      normal lines/.style={gray, very thin},
      every node/.append style={black, align=center, opacity=1}
    ]
    \foreach \y in {5.7,6.4,...,27.4}
        \draw[normal lines] (0,\y cm) -- (21cm,\y cm);
    \draw[draw=none] (3cm,0cm) -- (3cm,5.6cm);
    \draw[normal lines] (3 cm, 5.6 cm) -- (3 cm, 29 cm);
    \node (t) [font=\large, anchor=south, yshift=10pt] at ($(0,27.4)!1/2!(21cm,27.4)$) {Resposta da Questão Discursiva};
    \node (d) [font=\large, anchor=south west, xshift=8pt] at (0,26.85 cm) {\textbf{\questaoDiscursiva}};
    \end{tikzpicture}
  }
}

%
\vfill
\noindent
\begin{center}
    \textit{A tabela a seguir será preenchida pelo professor para fins de avaliação. Não escreva neste espaço.}
\end{center}
\vspace{3pt}
\begin{center}
    \begin{tabular}{C{150pt}|C{60pt}|C{60pt}|C{60pt}|C{60pt}|C{60pt}|}
        \cline{2-6}
        & \multicolumn{1}{c|}{\makecell{Muito bom\\100\%}}
        & \makecell{Bom\\75\%}
        & \makecell{Regular\\50\%}
        & \makecell{Ruim\\25\%} 
        & \makecell{Muito ruim\\0\%} \\
        
        \hline
            
        \multicolumn{1}{|c|}{\makecell{Aplicação da Matemática \\ 50\%}}
        & 
        & 
        & 
        & 
        &  \\

        \hline

        \multicolumn{1}{|c|}{\makecell{Organização e argumentação \\ 50\%}}
        & 
        & 
        & 
        & 
        &  \\

        \hline
    \end{tabular}
\end{center}
