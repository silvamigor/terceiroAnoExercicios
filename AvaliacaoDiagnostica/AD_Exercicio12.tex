%%%%%%%%%%%%%%%%%%%%%%%%%%%%%%%%%
%
%       EXERCÍCIO
%
%%%%%%%%%%%%%%%%%%%%%%%%%%%%%%%%%

\ifdefstring{\atividade}{prova}{%
    \ifdefstring{\modo}{objetivo}{%
        \renewcommand{\valorquestao}{\ValorQObj\ ponto}
    }{%
        \renewcommand{\valorquestao}{\ValorQDisc\ pontos}
    }%
}%

\begin{exercicioBanco}[\valorquestao]
O ponto \(P\) está representado no plano cartesiano a seguir.
%

\medskip

%
\begin{center}
    \begin{tikzpicture}[scale=0.5]
        % Eixos
        \draw[->] (-1,0) -- (5,0) node[right] {\(x\)};
        \draw[->] (0,-1) -- (0,5) node[above] {\(y\)};

        % Marcas nos eixos
        \foreach \x in {1,2,3,4}
            \draw (\x,0.1) -- (\x,-0.1) node[below] {\x};

        \foreach \y in {1,2,3,4}
            \draw (0.1,\y) -- (-0.1,\y) node[left] {\y};
        
        % Linhas tracejadas
        \draw[dashed] (3,0) -- (3,2) -- (0,2);
        
        % Ponto
        \fill (3,2) circle (2pt);
        \node[above right] at (3,2) {\(P\)};
    \end{tikzpicture}
\end{center}
%

\medskip

Quais são as coordenadas do ponto \(P\)?

% Define as alternativas
\newcommand{\alternativas}{%
    \begin{center}
        \begin{tabularx}{\textwidth}{XXXXX}
            (a) \((2,3)\). &
            (b) \((-3,2)\). &
            (c) \((3,-2)\). &
            (d) \((3,2)\). &
            (e) \((-2,3)\).
        \end{tabularx}
    \end{center}
}

% Define a resposta correta
\newcommand{\resposta}{D}

% Lógica condicional para exibição
\ifdefstring{\atividade}{lista}{%
    \alternativas
    \vspace{0.5em}
    
    \noindent\textbf{Resposta:} letra \textbf{\resposta}.
}{%
    \ifdefstring{\modo}{objetivo}{%
        \alternativas
    }{%
        % modo = discursiva → não mostra alternativas
    }
}
\end{exercicioBanco}

