%%%%%%%%%%%%%%%%%%%%%%%%%%%%%%%%%
%
%       EXERCÍCIO
%
%%%%%%%%%%%%%%%%%%%%%%%%%%%%%%%%%

\ifdefstring{\atividade}{prova}{%
    \ifdefstring{\modo}{objetivo}{%
        \renewcommand{\valorquestao}{\ValorQObj\ ponto}
    }{%
        \renewcommand{\valorquestao}{\ValorQDisc\ pontos}
    }%
}%

\begin{exercicioBanco}[\valorquestao]
A tabela abaixo mostra as temperaturas máximas registradas em uma cidade ao longo de cinco dias.

\medskip

\begin{center}
    \begin{tabular}{|c|c|}
        \hline
        \textbf{Dia} & \textbf{Temperatura (\(^{\circ}\)C)} \\
        \hline
        1 & 28 \\
        2 & 30 \\
        3 & 27 \\
        4 & 29 \\
        5 & 31 \\
        \hline
    \end{tabular}
\end{center}

\medskip

Qual foi a temperatura média registrada nesses cinco dias?

% Define as alternativas
\newcommand{\alternativas}{%
    \begin{center}
        \begin{tabularx}{\textwidth}{XXXXX}
            (a) \(28^{\circ}\)C. &
            (b) \(29^{\circ}\)C. &
            (c) \(30^{\circ}\)C. &
            (d) \(145^{\circ}\)C. &
            (e) \(27^{\circ}\)C.
        \end{tabularx}
    \end{center}
}

% Define a resposta correta
\newcommand{\resposta}{B}

% Lógica condicional para exibição
\ifdefstring{\atividade}{lista}{%
    \alternativas
    \vspace{0.5em}
    
    \noindent\textbf{Resposta:} letra \textbf{\resposta}.
}{%
    \ifdefstring{\modo}{objetivo}{%
        \alternativas
    }{%
        % modo = discursiva → não mostra alternativas
    }
}
\end{exercicioBanco}

