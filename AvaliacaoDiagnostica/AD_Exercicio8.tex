%%%%%%%%%%%%%%%%%%%%%%%%%%%%%%%%%
%
%       EXERCÍCIO
%
%%%%%%%%%%%%%%%%%%%%%%%%%%%%%%%%%

\ifdefstring{\atividade}{prova}{%
    \ifdefstring{\modo}{objetivo}{%
        \renewcommand{\valorquestao}{\ValorQObj\ ponto}
    }{%
        \renewcommand{\valorquestao}{\ValorQDisc\ pontos}
    }%
}%

\begin{exercicioBanco}[\valorquestao]
Na figura abaixo, os segmentos \(\overline{AB}\) e \(\overline{DE}\) são paralelos e os segmentos \(\overline{AE}\) e \(\overline{BD}\) se cruzam no ponto \(C\). Sabendo que \(DC = 3\) cm, \(AC = 4\) cm, \(CE = 6\) cm e \(BC = x\) cm. Determine o valor de \(x\).
%

\medskip

%
\begin{center}
    \begin{tikzpicture}[scale=0.8]

        % Pontos
        \coordinate (A) at (0,4);
        \coordinate (B) at (4,4);
        \coordinate (C) at (2,2.2);
        \coordinate (D) at (0,0);
        \coordinate (E) at (5,0);

        % Segmentos
        \draw (A) -- (B);
        \draw (A) -- (E);
        \draw (B) -- (D);
        \draw (D) -- (E);

        % Rótulos dos pontos
        \node[above left]  at (A) {A};
        \node[above right] at (B) {B};
        \node[left]        at (C) {C};
        \node[below left]  at (D) {D};
        \node[below right] at (E) {E};

        % Medidas
        \node[left, yshift=-5pt]  at ($(A)!0.5!(C)$) {4};
        \node[right, xshift=-1pt, yshift=-5pt] at ($(B)!0.5!(C)$) {$x$};
        \node[right, xshift=-5pt, yshift=10pt] at ($(C)!0.6!(E)$) {6};
        \node[left, xshift=5pt, yshift=10pt] at ($(C)!0.6!(D)$) {3};

    \end{tikzpicture}
\end{center}
%

\medskip

% Define as alternativas
\newcommand{\alternativas}{%
    \begin{center}
        \begin{tabularx}{\textwidth}{XXXXX}
            (a) \(2,5\). &
            (b) \(3\). &
            (c) \(4\). &
            (d) \(1,5\). &
            (e) \(2\).
        \end{tabularx}
    \end{center}
}

% Define a resposta correta
\newcommand{\resposta}{E}

% Lógica condicional para exibição
\ifdefstring{\atividade}{lista}{%
    \alternativas
    \vspace{0.5em}
    
    \noindent\textbf{Resposta:} letra \textbf{\resposta}.
}{%
    \ifdefstring{\modo}{objetivo}{%
        \alternativas
    }{%
        % modo = discursiva → não mostra alternativas
    }
}
\end{exercicioBanco}

